
\section{Experimental evaluation}
\label{sec:evaluation}
In this section we evaluate a Branch and Bound solver that uses our novel convex relaxation on point cloud registration and related outlier robust problems. We have implemented a basic version of BnB using best-first search. To branch over the spherical space of 3D rotations, we use rotation vector approach as in \cite{10.1007/978-3-642-37444-9_42}.

We also have validated the correctness of the algorithm and it's implementation empirically: We did not find any instance in which the lower bound was invalid, i.e. higher than the ground-truth global minimum. 

We perform the experiments on a desktop PC with an Intel i7 12700KF CPU and 32GiB of RAM.

\subsection{Translation experiments}
We evaluate solving the translation and the rotation problem separately. 

\subsection{Synthetic experiments}
\begin{figure}[!ht]
	\centering
%	\includegraphics[width=0.49\linewidth]{figures/execution_time_over_outlier_rate_boxplot}
%	\includegraphics[width=0.49\linewidth]{figures/rotation_error_over_outlier_rate.pdf}
	\caption{Execution time and rotation error over different outlier rates with $N=100$ on synthetic data.}
	\label{fig:execution-time-over-outlier-rate}
\end{figure}


\begin{figure}[!ht]
	\centering
	\caption{Rotation-only experiments: The average rotation error over the outlier rate: The Truncated Least Squares method is robust up to 95\% outliers.}
	\label{fig:rotation-error-over-outlier-rate}
\end{figure}


\begin{figure}[!ht]
	\centering
	\caption{Average runtime of our Branch-and-Bound solver depending on the total amount of point correspondences $N$.}
	\label{fig:runtime}
\end{figure}


We generate synthetic point clouds by sampling inliers from a unit cube $s \cdot [-1, 1]^3$ with scale $s$. We add uniform measurement noise by sampling a ball of radius $\epsilon^2$, and then sample outliers in the bounding box of the inliers, also uniformly. First, we set $s=10$ and $\epsilon=0.5$. As shown in Fig. \ref{fig:execution-time-over-outlier-rate}, the runtime is very low for $N =100$.

The results of the rotation error as a function of the outlier rate are shown in Fig. \ref{fig:rotation-error-over-outlier-rate} is shown. The estimation remains correct with a small error of less than $1$ degree. We also investigated the scalability of our method over the total number of points $N$. As the results in Fig. \ref{fig:runtime} show, our approach can even be used for real-time applications with a runtime of less than 100 milliseconds.

\subsubsection{Scalability}
\begin{figure}[!ht]
	\centering
	%\includegraphics[width=0.9\linewidth]{figures/scalability-synth}
	\caption{Scalability on synthetic data, outlier rate $75\%$}
	\label{fig:execution-time-over-n}
\end{figure}

We evaluate the scalability on synthetic data at $75\%$ outlier-rate, results are shown in Fig \ref{fig:execution-time-over-n}. Our solves is efficient since it implicitly prunes vertices based on the TLS objective.

\subsection{Rotation experiments}
We solve the rotation-only truncated least squares problem with the same approach as the translation problem: We use the same BnB solver and convex relaxation, only the computation of the residual intervals changes, as well as the minimization of the relaxation function.


\subsubsection{Validity of the global optimality certificate}
We test on synthetic data whether the global optimality certificate is valid, i.e. whether at termination, the lower bound is less or equal than the TLS objective at the ground-truth solution.
We define the suboptimality $\eta$ of the solution based on the upper bound (UB), that is, the best known solution, and the lower bound (LB) \cite{9785843}
\begin{equation}
	\begin{aligned}    
		\eta = \frac{\text{UB} - \text{LB}}{1 + |\text{UB}| + |\text{LB}|}
	\end{aligned}
\end{equation}
We test with $N=100$, $\epsilon=0.5$ and vary the data scale $s$ from 3 to 100. We test for each scale variation the outlier rates $0\%, 30\%, 50\%, 90\%, 96\%$. We do 1000 trials for each such parameter set.

\begin{figure}[!ht]
	\centering
	%\includegraphics[width=0.49\linewidth]{figures/certificate_validity_vs_scale}
	%\includegraphics[width=0.49\linewidth]{figures/eta_subopt_over_outlier_rate}
	\caption{Amount of instances with valid global optimality certificate (i.e. valid suboptimality) on synthetic data, including achieved $\eta$-suboptimality.}
	\label{fig:cert-validity}
\end{figure}

TODO 

\subsubsection{Comparison with SDP-relaxation solver STRIDE} Currently the only solver that can solve the TLS registration problem of size $N=100$ to proven global optimality is STRIDE \cite{9785843}. We compare it on our synthetic data to access the achieved suboptimality and estimation errors. We use the public MATLAB-implementation of the authors and use the MOSEK solver for the chordal initialization of STRIDE.
We set $\epsilon = 0.5$, $N=30$ and perform three trials per outlier rate.

\begin{figure}[!ht]
	\centering
	%\includegraphics[width=0.49\linewidth]{figures/subopt_stride}
	%\includegraphics[width=0.49\linewidth]{figures/runtime_stride}
	\caption{The reached suboptimality and the runtime using STRIDE over the outlier rate on synthetic data, $N=30$.}
	\label{fig:stride-subopt-eval}
\end{figure}
\begin{figure}[!ht]
	\centering
	%\includegraphics[width=0.9\linewidth]{figures/rotation_error_stride}
	\caption{Rotation error using STRIDE.}
	\label{fig:stride-rot-err}
\end{figure}

The results are shown in Fig. \ref{fig:stride-subopt-eval}, STRIDE is able to find and certify the global optimum as long as the outlier rate is below 90\%. The runtime of STRIDE is even with as few as 30 points in the order of $10^3$ seconds. In instances that STRIDE could solve (i.e. where the relaxation is tight), it achieves a similar rotation error compared to our solver (Fig. \ref{fig:stride-rot-err}). Note that the rotation error is meaningless for instances that STRIDE could not solve (i.e. above 80\% outliers).

\subsection{Results on 3DMatch dataset}
  \begin{table}[!ht]
  	\centering
  	\caption{Registration accuracy on the 3DMatch dataset.\label{tab:3DMatch-results}}
  	\resizebox{\linewidth}{!}{
  		\begin{tabular}{l|rrr|rrr}
  			\hline
  			\multirow{2}[2]{*}{} Descriptor & \multicolumn{3}{c|}{FPFH} & \multicolumn{3}{c}{FCGF} \\
  			& \multicolumn{1}{l}{RR(\%)} & \multicolumn{1}{l}{RE(\textdegree)} & \multicolumn{1}{l|}{TE(cm)} & \multicolumn{1}{l}{RR(\%)} & \multicolumn{1}{l}{RE(\textdegree)} & \multicolumn{1}{l}{TE(cm)} \\
  			\hline  			
  			RANSAC-1M & 64.20 & 4.05 & 11.35 & 88.42 & 3.05 & 9.42 \\
  			RANSAC-4M & 66.10 & 3.95 & 11.03 & 91.44 & 2.69 & 8.38 \\
  			TEASER++\cite{Yang20tro-teaser} & 75.48 & 2.48 & 7.31 & 85.77 & 2.73 & 8.66 \\
  			CG-SAC\cite{quan2020compatibility} & 78.00 & 2.40 & 6.89 & 87.52 & 2.42 & 7.66 \\
  			$\rm{SC}^2$-PCR\cite{SC2-PCR-Chen-2022-CVPR} & 83.73  & 2.18  & {6.70} & 93.16  & \underline{2.09}  & 6.51 \\
  			PointDSC\cite{bai2021pointdsc} & 72.95 & 2.18  & 6.45  &  91.87 & 2.10  & {6.54} \\
  			MAC \cite{zhang20233d} & 84.10  & 1.96  & 6.18  &  93.72 &  1.89  & 6.03 \\
  			\hline
  			\MyAlgo{} (Ours) & 54.84 & 3.27 & 9.29 & 83.92 & 2.29 & 7.14 \\
  			\hline
  	\end{tabular}}
  \end{table}%
We evaluated our solver on the 3DMatch dataset \cite{zeng20163dmatch} 
following the same protocol as in \cite{SC2-PCR-Chen-2022-CVPR, zhang20233d}. We use all scenes and evaluate the residual rotation error (RE) and translation error (TE) as well as the registration recall (RR). A registration is defined as successful if RE $\leq 15$ degrees and TE $\leq 30$ cm holds. We sub-sample the correspondences to 500 and set the inlier-threshold to $\epsilon = 0.6$. For the other methods, we report the results from \cite{zhang20233d} since the test setup is identical. The results are shown in Tab. \ref{tab:3DMatch-results}.

\subsection{Results on KITTI dataset}
\begin{table}[!ht]
    \centering
    \caption{Registration accuracy on KITTI dataset in comparison with other methods:\label{tab:eval-kitti}}
    \resizebox{\linewidth}{!}{
      \begin{tabular}{l|rrr|rrr}
      \hline
      \multirow{2}[2]{*}{} Descriptor& \multicolumn{3}{c|}{FPFH} & \multicolumn{3}{c}{FCGF} \\
            & \multicolumn{1}{l}{RR(\%)} & \multicolumn{1}{l}{RE(\textdegree)} & \multicolumn{1}{l|}{TE(cm)} & \multicolumn{1}{l}{RR(\%)} & \multicolumn{1}{l}{RE(\textdegree)} & \multicolumn{1}{l}{TE(cm)} \\
      \hline
      TEASER++\cite{Yang20tro-teaser}   &  91.17  &  1.03  & 17.98 & 94.96   &  0.38  & 13.69 \\
      RANSAC  &  74.41  &  1.55  &  30.20 & 80.36  &  0.73  & 26.79 \\
      CG-SAC\cite{quan2020compatibility}  & 74.23  & 0.73  & 14.02 & 83.24  & 0.56  & 22.96 \\
      $\rm{SC}^2$-PCR\cite{SC2-PCR-Chen-2022-CVPR} & 99.64  &  0.32  & 7.23  & 98.20 & 0.33  & 20.95 \\
      MAC \cite{zhang20233d} &  99.46  &  0.4  & 8.4  &  97.84 &  0.34  & 19.34 \\
      \hline
      \MyAlgo{} (Ours) &  95.50 & 0.77  & 16.43  &  96.76 & 0.34 & 20.59 \\
      \hline
      \end{tabular}}
  \end{table}%

We evaluate our method on the KITTI odometry dataset \cite{Kitti-Geiger2013IJRR} following \cite{SC2-PCR-Chen-2022-CVPR, zhang20233d} and similar to the 3DMatch dataset. A registration is defined as successful if RE $\leq 5$ degrees and TE $\leq 60$ cm.
We sub-sample the correspondeces to 500 and set the inlier-threshold to $\epsilon = 0.6$. We set a time limit of 100 milliseconds. The FCGF \cite{9009829} descriptor has on average an outlier-rate of 75\%, while the FPFH has usually above 90\% outlier-rate. For the other methods, we report the results from \cite{zhang20233d} since the test setup is identical. The results are shown in Tab. \ref{tab:eval-kitti}, our method achieves comparable results to other state-of-the-art methods without dataset-specific tuning of parameters, outperformed only by MAC and SC$^2$-PCR.


